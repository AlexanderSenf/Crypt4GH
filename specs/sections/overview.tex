\section{Encrypted Representation Overview}
The encrypted file consists of five parts:

\begin{itemize}
\item An unencrypted header, containing a magic number, a version number, an optional public key along with its length (potentially zero), and the length of the encrypted header.
%
\item An encrypted header, which is encrypted for individual recipients using an asymmetric key scheme. 
%
It lists the checksum algorithm used, the encryption algorithm used, and the encryption key needed to decrypt the encrypted data portion.
%
\item The encrypted data.
% 
This is the actual application data.
%
It is encrypted using a symmetric encryption algorithm, as specified in the encrypted header.
%
The data is encrypted in 64 kilobytes segments. For each encrypted segment, a 12 byte nonce is prepended and a 16 byte MAC is appended.
%
\item The checksum of the unencrypted data. If a checksum algorithm other than 'none' was listed in the encrypted header, the hexadecimal representation of the checksum over the unencrypted data is appended to the (unencrypted) data.
\end{itemize}
