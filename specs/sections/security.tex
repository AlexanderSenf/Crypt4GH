\section{Security Considerations}
\subsection{Threat Model}
This format is designed to protect files at rest and in transport from
accidental disclosure.
%
Using authenticated encryption in individual segments mirrors
solutions like Transport Layer Security (TLS) as described
in~\cite{RFC5246} and prevents undetected modifications of segments.

\subsection{Selection of Key and Nonce}
The security of the format depends on attackers not being able to
guess the encryption key (and to a lesser extent the nonces).
%
The encryption key and nonce MUST be generated using a
cryptographically-secure pseudo-random number generator.
%
This makes the chance of guessing a key vanishingly small.
%
Additional security can be provided by using `Associated Data' when
encrypting a file.
%
This data must be used to decrypt the data, although it is not part of
the encrypted file \cite{RFC8439}.

\subsection{Message Forgery}
The encrypted header part is authenticated using the Ed25519 signature
scheme.
%
Using ChaCha20-ietf-Poly1305, the Poly1305 algorithm produces an
authentication tag for each encrypted segment.

\subsection{No File Updates Permitted}
Implementations MUST NOT update encrypted files.
%
Once written, a section of the file must never be altered.
